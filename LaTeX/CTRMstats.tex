\documentclass[12pt]{article}

\usepackage{amsmath,amssymb,amsthm}
%\usepackage{amsfonts,dsfont,mathrsfs}
%\usepackage[mathscr]{eucal}
\usepackage{graphicx}
\usepackage{color}
\pagestyle{plain}
\usepackage[margin=30mm]{geometry}

\usepackage{lineno} %for line numbers
%\usepackage{showkeys}
\usepackage{color}

\usepackage{enumerate}
\usepackage{url}
\usepackage{hyperref}
% \usepackage{breakurl}
\definecolor{darkblue}{rgb}{0,0,.75}
\definecolor{darkgreen}{rgb}{0,0.5,0}
\hypersetup{colorlinks=true, breaklinks=true, linkcolor=blue, citecolor=darkblue, menucolor=darkblue, urlcolor=darkblue}
%pdftex=true,

%\usepackage[round]{natbib}
\usepackage{doi}


% \usepackage{helvet}
% \usepackage{sfmath}
% \usepackage{sansmathfonts}

% \renewcommand\familydefault{\sfdefault}

%\usepackage{arev}


\newtheorem{theorem}{Theorem}[section]
\newtheorem{lemma}[theorem]{Lemma}
\newtheorem{proposition}[theorem]{Proposition}
\newtheorem{cor}[theorem]{Corollary}
\newtheorem{corollary}[theorem]{Corollary}
\newtheorem{definition}[theorem]{Definition}

\theoremstyle{definition}
\newtheorem{defi}[theorem]{Definition}
\newtheorem{example}[theorem]{Example}

\theoremstyle{remark}
\newtheorem{remark}[theorem]{Remark}

\numberwithin{equation}{section}

\newcommand{\ex}{\mathbf {E}}
\newcommand{\pr}{\mathbf {P}}
\newcommand{\R}{\mathbb R}
\newcommand{\Rd}{\mathbb R^d}
\newcommand{\Rp}{\mathbb R^+}
\newcommand{\spctim}{\mathbb R^{d+1}}
\newcommand{\del}{\partial }
\newcommand{\1}{\mathbf 1}
\newcommand{\eps}{\varepsilon}

\begin{document}

%\linenumbers

\title{CTRM Statistics}
\maketitle
\begin{abstract}
We fit a max-renewal process, (aka 'CTRM', Continuous Time Random Maxima)
to data. 

\end{abstract}

%60F17  	Functional limit theorems; invariance principles
%60B10  	Convergence of probability measures
%60G22  	Fractional processes, including fractional Brownian motion
%60G50  	Sums of independent random variables; random walks

\section{Introduction}

Time series displaying inhomogeneous behaviour have received strong interest in 
the recent statistical physics literature,
\cite{Barabasi2005,Oliveira2005,Vasquez2006,Vazquez2007,Omi2011,
Min2010,Karsai2011,Bagrow2013},
and have been observed in the context of earthquakes, sunspots, neuronal
activity, human communication etc., see \cite{Karsai2012,Vajna2013} for a 
list of references.
Such time series exhibit high activity in some `bursty' intervals, which 
alternate with other, quiet intervals.  Although several mechanisms are 
plausible explanations for bursty behaviour
(most prominently self-exciting point processes \cite{hawkes1971point}),
there seems to be one salient
feature which very typically indicates the departure from temporal homogeneity: 
A heavy-tailed distribution of waiting times
\cite{Vasquez2006,Karsai2012,Vajna2013}. 
A simple renewal process with heavy-tailed waiting times captures these
dynamics. For many systems, the renewal property is appropriate, as can be
checked by a simple test: the dynamics do not change significantly if the
waiting times are randomly reshuffled \cite{Karsai2012}.

When a magnitude can be assigned to each event in the renewal process, 
such as e.g.\
for earthquakes, sun flares or neuron voltages,
two natural and basic questions to ask are: 
What is the distribution of the largest event up to a given time $t$?
What is the probability that an event exceeds a given level $\ell$ within the
next $t$ units of time?
A probabilistic extreme value model which assumes that the events form a 
renewal process is available in the literature. 
This model has been studied under the names
``Continuous Time Random Maxima process'' (CTRM) 
\cite{Benson2007,MeerschaertStoev08,Hees16,Hees14}, 
``Max-Renewal process'' \cite{Silvestrov2002a,ST04,Basrak2014}, 
and ``Shock process'' 
\cite{Esary1973,Sumita1983,Sumita1984,Sumita1985,Anderson1987,Gut1999}.
This article aims to develop concrete statistical inference methods for this model, 
a problem which has seemingly received little attention by the statistical 
community. 

\section{CTRM processes}

The Continuous Time Random \emph{Walk} (CTRW) has been a highly successful 
model for anomalous diffusion in the past two decades 
\cite{Metzler2000,HLS2010b}, likely due to its tractable and flexible scaling
properties.
The stochastic process we study in this article is conceptually very close
to the CTRW, since essentially the jumps $J_k$ are reinterpreted as 
magnitudes, and instead of the cumulative sum, one tracks the cumulative 
maximum. Similarly tractable scaling properties apply to the CTRM, see below.
For the above reasons, we use the name CTRM in this article.


\begin{definition}
Assume i.i.d.\ pairs of random variables $(J_k, W_k)$, $k = 1, 2, \ldots$
where $W_k > 0$ represents the inter-arrival times of certain events and 
$J_k \in \R$ the corresponding event magnitudes.  
Write 
\begin{align} \label{eq:renewal-process}
N(t) = \max\{n \in \mathbb N: W_1 + \ldots + W_n \le t\}
\end{align}
for the renewal process
associated with the $W_k$ (where the maximum of the empty set is set to $0$).
Then the process
\begin{align}
M(t) = \bigvee_{k=1}^{N(t)} J_k
= \max\{J_k: k = 1, \ldots, N(t)\}, \quad t \ge 0.
\end{align}
is called a CTRM (Continuous Time Random Maxima process).
\end{definition}

 

It is clear that the sample paths of $N(t)$ and $M(t)$ are right-continuous
with left-hand limits. 
If $W_k$ is interpreted as the time \emph{leading up} to the event with magnitude
$J_k$, then $M(t)$ is indeed the largest magnitude up to time $t$.
The alternative case where $W_k$ represents the inter-arrival time 
\textit{following} $J_k$ 
is termed ``second type'' (in the shock model literature) or OCTRM
(overshooting CTRM), and the largest magnitude up to time $t$ is then
given by
\begin{align}
\tilde M(t) = \bigvee_{k=1}^{N(t)+1} J_k, \quad t \ge 0.
\end{align}
Finally, the model is called \emph{coupled} when $W_k$ and $J_k$ are not independent.
In this article we focus on the uncoupled case,  
for which it can be shown that the processes $M(t)$ and
$\tilde M(t)$ have the same limiting distributions at large times
\cite{Hees16}, and hence we focus on the CTRM $M(t)$. 


We aim to make inference on the distribution of the following quantities:
\begin{definition}
Let $M(t)$ be an uncoupled CTRM whose magnitudes $J_k$ are supported on the interval
$[x_0, x_F]$.  Then the exceedance time of level $\ell \in [x_0,x_F]$ is
the random variable
\begin{align*}
T_\ell = \inf\{t: M(t) > \ell\}
\end{align*}
and the exceedance is 
\begin{align*}
X_\ell = M(T_\ell) - \ell.
\end{align*}
\end{definition}

\begin{lemma}
Given a level $\ell \in [x_0,x_F]$, exceedance $X_\ell$ and exceedance time
$T_\ell$ are independent. Moreover, 
\begin{align*}
\pr[X_\ell > x]
&= \frac{\overline F_J(\ell + x)}{\overline F_J(\ell)}, \quad x > 0,
\\
\pr[T_\ell > t]
&= \overline F_J(\ell) \sum_{n=1}^\infty \int_0^\infty \overline F_W(t-t') \pr[L_{n-1} \le \ell]
\pr[S_{n-1} \in dt'], \quad t > 0
\end{align*}
where $L_n = \bigvee_{k=1}^n J_k$, $L_0 = x_0$, and $S_n = \sum_{k=1}^n W_k$,
$S_0 = 0$.
\end{lemma}

\begin{proof}
Let $\tau_\ell = \min\{k: J_k > x\}$. Then
\begin{align*}
T_\ell > t, X_\ell > x 
&\Longleftrightarrow
S_{\tau_\ell} > t, L_{\tau_\ell} > \ell + x
\Longleftrightarrow
\exists n: L_{n-1} \le \ell, J_n > \ell + x, S_n > t
\\
&\Longleftrightarrow
\exists n: L_{n-1} \le \ell, J_n > \ell + x, W_n > t - S_{n-1}
\end{align*}
and such $n$ must be unique, since $x > 0$.
Thus 
\begin{align*}
&\pr[T_\ell > t, X_\ell > x]
= \sum_{n=1}^\infty \pr[J_n > \ell + x, W_n > t - S_{n-1}, L_{n-1} \le \ell]
\\
&= \sum_{n=1}^\infty \int_0^\ell \int_0^\infty \pr[J_n > \ell + x, W_n > t - t' | L_{n-1} = m', 
S_{n-1} = t'] \pr[L_{n-1} \in dm', S_{n-1} \in dt']
\\
&= \sum_{n=1}^\infty \int_0^\infty \overline F_J(\ell + x) \overline F_W(t - t')
\pr[L_{n-1} \le \ell] \pr[S_{n-1} \in dt']
\end{align*}
since the sequences $J_k$ and $W_k$ are i.i.d.\ and independent of each other.
Letting $t \downarrow 0$, we see
\begin{align*}
&\pr[X_\ell > x] 
= \sum_{n=1}^\infty \overline F_J(\ell+x) \pr[L_{n-1} \le \ell] 
= \overline F_J(\ell+x) \sum_{n=1}^\infty F_J(\ell)^{n-1}
\\
&= \frac{\overline F_J(\ell + x)}{\overline F_J(\ell)},
\end{align*}
and letting $x \downarrow 0$, one gets $\pr[T_\ell > t]$. 
One checks that $\pr[T_\ell > t, X_\ell > x] = \pr[T_\ell > t] \pr[X_\ell > x]$, implying independence.
\end{proof}
The distribution of the exceedance is hence, as expected, simply the
conditional distribution of a magnitude $J_k$ given $J_k > \ell$,
and may hence be modeled by a generalized Pareto (GP) distribution
\cite{ColesBook}. The exceedance times, and hence the temporal aspect of the
dynamics, are independent of the exceedances, and may be looked at separately.

\section{Fractional Time Scale}

%By the stable central limit theorem, it is well known that for a large 
%number of observation pairs $(J_k,W_k)$, 
%%the maximum of $J_k$
%%approaches a max-stable (GEV) distribution \cite{ColesBook}, whereas 
%the distribution of the sum of $W_k$ approaches 
%a sum-stable distribution \cite{MeerschaertSikorskii}.  
As mentioned in the introduction, we assume a heavy-tailed distribution 
for the waiting times, more precisely that the tail function 
$\overline F_W(x) := 1 - F_W(x)$ of the CDF of $W$ is regularly varying:
\begin{align}
t \mapsto \overline F_W(t) \in RV(-\beta), \quad \beta \in (0,1)
\end{align}
meaning that \cite{seneta,thebook}
\begin{align*}
\lim_{t \to \infty}\frac{\overline F_W(\lambda t)}{\overline F_W(t)}
= \lambda^{-\beta}.
\end{align*}
By \cite[Cor.~8.2.19]{thebook} (also compare \cite[Th.~4.5.1]{Whitt2010}), this is equivalent to $W_1$ being in the 
(sum-) domain of attraction of a positively skewed stable random variable $D$ with 
stability parameter $\beta$, which means the weak convergence
\begin{align}
(W_1 + \ldots + W_n)/b(n) \Rightarrow D, \quad (n \to \infty)
\end{align}
for some $b(n) \in RV(1/\beta)$.
The i.i.d.\ nature of the $W_k$ allows for a more general, functional limit
theorem \cite[Ex.~11.2.18]{thebook}, \cite[Th.~4.5.3]{Whitt2010}:
\begin{align}
(W_1 + \ldots + W_{\lfloor cr \rfloor}) / b(c) \Rightarrow D(r), 
\quad (c \to \infty),
\end{align}
where $\{D(r)\}_{t \ge 0}$ is a stable subordinator, defined via 
\begin{align}
\ex[e^{-\lambda D(r)}] = e^{-r \lambda^\beta},
\end{align}
where $\lfloor r \rfloor$ denotes the largest integer not larger than $r$,
and where convergence holds on the stochastic process level, i.e.\ in the sense
of weak convergence in Skorokhod space endowed with the $J_1$ topology
\cite[Sec.~3.3]{Whitt2010}. 

The renewal process paths $N(t)$ result, up to a constant, from
inverting the non-decreasing sample paths of the sum
$r \mapsto W_1 + \ldots + W_{\lfloor r \rfloor}$.
Accordingly, the scaling limit of the renewal process is \cite{limitCTRW} 
\begin{align}
N(ct) / \tilde b(c) \Rightarrow E(t), \quad c \to \infty
\end{align}
where $E(t)$ denotes the inverse stable subordinator \cite{invSubord}
\begin{align}
E(t) = \inf\{r: D(r) > t\}, \quad t \ge 0,
\end{align}
and where $\tilde b(c)$ is asymptotically inverse to $b(c)$, in the sense
of \cite[p.20]{seneta}: 
\begin{align}
b(\tilde b(c)) \sim c \sim \tilde b(b(c))
\end{align}
where the $\sim$ symbol indicates that the quotient of both sides converges to
$1$ as $c \to \infty$. 
Note that $\tilde b(c) \in RV(\beta)$. 

The inverse stable subordinator $E(t)$ governs the temporal dynamics of
the scaling limit of the CTRM process $M(t)$, see Theorem~\ref{th:AEt} below. 
It is self-similar with exponent $\beta$
\cite{limitCTRW}, non-decreasing, and the (regenerative, random) set 
$\mathcal R$ of its points of increase is a fractal with dimension $\beta$ 
\cite{Bertoin04}.
$E(t)$ is a model for time series with intermittent, `bursty'
behaviour, for the following reasons:
\begin{itemize}
\item [i)]
Conditional on $t \ge 0$ being a point of increase, any interval 
$(t, t+ \epsilon)$ almost surely contains uncountably many other points of 
$\mathcal R$; 
\item [ii)]
$\mathcal R$ has Lebesgue measure $0$, and hence $E(t)$ is
constant at any ``randomly'' chosen time $t$. 
\end{itemize}
Having only two parameters ($\beta \in (0,1)$ and a scale parameter)
the inverse stable subordinator hence models scaling limits of heavy-tailed 
waiting times parsimoniously.

\section{Generalized Extreme Value Distributions}

We turn our attention to the exceedances $X_\ell$ of a given high threshold
$\ell$ `close' to the right end-point $x_F$ of $F_J$. 
Any non-degenerate distribution on $\R$

\section{CTRM limit}

\paragraph*{The timing of maxima.}
The distribution of the largest magnitudes is well approximated by a 
generalized extreme value (GEV) distribution $F$ if the number of events is 
large \cite{ColesBook}.
Consider again the light-tailed inter-arrival time case: 
Exceedances of large levels $\ell$ with small occurrence probability
$p = 1 - F(\ell)$ occur, on average every $1/p$ observations, and the average
return period is quickly seen to be $\ex[W_k]/p$.
This provides an intuitive interpretation of magnitudes, e.g.\ ``a flood
exceeding level $\ell$ occurs only every 100 years on average.'' 
Moreover, the lightness of the tails of $W_k$ implies, as mentioned above,
that events occur in a Poisson process fashion.  The memory of large events
in the past is negligible for the prediction of large events in the future.
For heavy-tailed inter-arrival times, however, the Markov property does not
hold, and time averages of return periods are not defined, since their moments
are infinite.  


\paragraph*{Aims.}
The goal of our statistical method is to make inferences about the dynamics
of $M(t)$, given data $(W_k, J_k)$. 
Since the waiting times $J_k$ are not assumed exponential, $M(t)$ is not Markov,
and the same applies to $A(E(t))$.  



As was shown by \cite{Anderson1987} (see also 
\cite{MeerschaertStoev08}), exceedance times are
Mittag-Leffler distributed, and thus form a \emph{fractional} Poisson process
\cite{Laskin2003}. Due to recent progress in the estimation of Mittag-Leffler
distributions \cite{Cahoy2013,Cahoy2010} we are able to model the recurrence
periods, and thus to give predictive distributions for future exceedance times,
using the semi-Markov property of fractional Poisson processes.



\cite{MeerschaertStoev08} have shown that for large times $t$, 
the process $M(t)$ converges to a (rescaled) time-changed 
extreme-value process $A(E(t))$. 
Here, $A(t)$ is an extreme-value process in the sense of \cite{Lamperti1964}
and \cite{resnick2013extreme}, and 









Any probability distribution with cumulative distribution
function (CDF) $F$ on $\R$ lies in the max-domain of attraction
of a generalized extreme value (GEV) distribution with CDF $\Lambda$
\cite{beirlant06Book}.
That is, there exist
functions $a: \Rp \to \Rp$ and $d: \Rp \to \Rp$ such that
\begin{align}
F(a(n) x + b(n))^n \to \Lambda(x), \quad n \to \infty, \quad x \in (x_L, x_R),
\end{align}
where $x_L$ and $x_R$ denote the left and right endpoint of the GEV distribution.



\section{Statistical Method}

\section{Simulated Data}

\section{Real Data}

\section{Conclusion}

\bibliographystyle{alpha}

\bibliography{/Users/poyda/BibTex/CTRMstats}

\end{document}
