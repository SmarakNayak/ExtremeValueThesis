\documentclass[12pt]{amsart}

\usepackage{amsmath,amssymb,amsthm}
%\usepackage{amsfonts,dsfont,mathrsfs}
%\usepackage[mathscr]{eucal}
\usepackage{graphicx}
\usepackage{color}
\pagestyle{plain}
\usepackage[margin=30mm]{geometry}

\usepackage{lineno} %for line numbers
%\usepackage{showkeys}
\usepackage{color}

\usepackage{enumerate}
\usepackage{url}
\usepackage{hyperref}
\usepackage{breakurl}
\definecolor{darkblue}{rgb}{0,0,.75}
\definecolor{darkgreen}{rgb}{0,0.5,0}
\hypersetup{colorlinks=true, breaklinks=true, linkcolor=blue, citecolor=darkblue, menucolor=darkblue, urlcolor=darkblue}
%pdftex=true,

%\usepackage[round]{natbib}
\usepackage{doi}


% \usepackage{helvet}
% \usepackage{sfmath}
% \usepackage{sansmathfonts}

% \renewcommand\familydefault{\sfdefault}

%\usepackage{arev}


\newtheorem{theorem}{Theorem}[section]
\newtheorem{lemma}[theorem]{Lemma}
\newtheorem{proposition}[theorem]{Proposition}
\newtheorem{cor}[theorem]{Corollary}
\newtheorem{corollary}[theorem]{Corollary}

\theoremstyle{definition}
\newtheorem{defi}[theorem]{Definition}
\newtheorem{example}[theorem]{Example}

\theoremstyle{remark}
\newtheorem{remark}[theorem]{Remark}

\numberwithin{equation}{section}

\newcommand{\ex}{\mathbf {E}}
\newcommand{\pr}{\mathbf {P}}
\newcommand{\R}{\mathbb R}
\newcommand{\Rd}{\mathbb R^d}
\newcommand{\Rp}{\mathbb R^+}
\newcommand{\spctim}{\mathbb R^{d+1}}
\newcommand{\del}{\partial }
\newcommand{\1}{\mathbf 1}
\newcommand{\eps}{\varepsilon}

\author{ }
\address{}
\email{}
\address{Peter Straka, School of Mathematics and Statistics,
UNSW Australia,
Sydney, NSW 2052, Australia}
\email{p.straka@unsw.edu.au}

%\linespread{1.5}


\begin{document}



%\linenumbers

\title{CTRM Statistics}
%\subtitle{Limit Theorems and Governing Equations}


%\titlerunning{Short form of title}   % if too long for running head

%\authorrunning{Short form of author list} % if too long for running head


%\date{Received: date / Accepted: date}
% The correct dates will be entered by the editor




\begin{abstract}
We fit a max-renewal process, (aka 'CTRM', Continuous Time Random Maxima)
to data. 
%\textbf{Keywords:} anomalous diffusion;
%fractional kinetics;
%fractional derivative;
%subordination;
%coupled random walks
%\\
%\keywords{anomalous diffusion
%\and functional limit theorem
%\and fractional derivative
%\and subordination
%\and coupled random walks
%\and fractional kinetics}
%% \PACS{PACS code1 \and PACS code2 \and more}
%\textbf{2010 MSC:} 60F17; 60G22%;60B10 \and 60G50}
\end{abstract}
\maketitle

%60F17  	Functional limit theorems; invariance principles
%60B10  	Convergence of probability measures
%60G22  	Fractional processes, including fractional Brownian motion
%60G50  	Sums of independent random variables; random walks



\section{Introduction}

%% CTRMs; the process
The stochastic model $M(t)$ below has been studied under the various names
``Shock process''
\cite{Esary1973,Sumita1983,Sumita1984,Sumita1985,Anderson1987,Gut1999}, 
``max-renewal process'', 
\cite{Silvestrov2002a,ST04,Basrak2014}
and ``CTRM (continuous time random maxima)
\cite{Benson2007,Meerschaert2009,Hees16,Hees2015}: 
Assume i.i.d. pairs of random variables $(W_k, J_k)$, $k = 1, 2, \ldots$
where $W_k > 0$ represents an inter-arrival time of certain events and 
$J_k \in \R$ the corresponding event magnitude.  
Write $N(t) = \max\{n \in \mathbb N: W_1 + \ldots + W_n \le t\}$ for the renewal process
associated with the $W_k$ (where the maximum of the empty set is set to $0$), 
and define
\begin{align}
M(t) = \bigvee_{k=1}^{N(t)} J_k
= \max\{J_k: k = 1, \ldots, N(t)\}.
\end{align}
If $W_k$ is interpreted as the time leading up to the event with magnitude
$J_k$, then $M(t)$ is the largest magnitude up to time $t$.
The alternative case where $W_k$ is the inter-arrival time \textit{following} $J_k$ 
is termed ``second type'' (in the shock model literature) or OCTRM
(overshooting CTRM), and the largest magnitude up to time $t$ is then
given by
\begin{align}
\tilde M(t) = \bigvee_{k=1}^{N(t)+1} J_k.
\end{align}
Finally, the model is called coupled when $W_k$ and $J_k$ are not independent;
in this article we focus on the uncoupled case.  For this case, $M(t)$ and
$\tilde M(t)$ have the same limiting behaviour for large times and magnitudes
\cite{Hees2015} and hence we focus on the CTRM $M(t)$. 

%% Standard case
If $J_k$ is constant or light-tailed (e.g.\ exponentially distributed), 
then, by the strong law of large numbers,
on large time scales the number $N(t)$ of events by time $t$ is well
approximated by $t/\ex[W_k]$. Magnitudes in excess of a given (large)
threshold $\ell$ then arrive at a constant rate (which depends on $\ell$), 
and the inter-arrival times are exponentially distributed. 
The dynamics of the record process are well known.

%% Heavy tailed case, temporal evolution. Self-similarity, fractal, bursts
In this article, we are concerned with heavy-tailed inter-arrival times $W_k$: 
We assume that $\pr[W_k > t] \sim C t^{-\beta}$ as $t \to \infty$, 
for some $\beta \in (0,1)$ and some constant $C$. 
Then $N(t)$ has temporal scaling exponent $\beta$, i.e.\
$C n^{-\beta} N(nt) \to E(t)$, where 
$E(t)$ is the inverse of a stable subordinator
(increasing L\'evy process) $D(t)$, in the sense of 
\begin{align}
E(t) = \inf\{u: D(u) > t\}.
\end{align}
The process $E(t)$ is self-similar, and its points of increase model `bursts',
meaning that $E(t)$ is starkly increasing in some short time intervals and
constant in others \cite{Karsai2012}.


%% Limit theorem for M(t)
\cite{Meerschaert2009} have shown that for large times $t$, 
the process $M(t)$ converges to a (rescaled) time-changed 
extreme-value process $A(E(t))$. 
Here, $A(t)$ is an extreme-value process in the sense of \cite{Lamperti1964}
and \cite{resnick2013extreme}, and 


%% GOAL
The goal of our statistical method is to make inferences about the dynamics
of $M(t)$, given data $(W_k, J_k)$. 
Since the waiting times $J_k$ are not assumed exponential, $M(t)$ is not Markov,
and the same applies to $A(E(t))$.  








Any probability distribution with cumulative distribution
function (CDF) $F$ on $\R$ lies in the max-domain of attraction
of a generalized extreme value (GEV) distribution with CDF $\Lambda$
\cite{leadbetter2012extremes}.
That is, there exist
functions $a: \Rp \to \Rp$ and $d: \Rp \to \Rp$ such that
\begin{align}
F(a(n) x + b(n))^n \to \Lambda(x), \quad n \to \infty, \quad x \in (x_L, x_R),
\end{align}
where $x_L$ and $x_R$ denote the left and right endpoint of the GEV distribution.






\bibliographystyle{alpha}

\bibliography{../../../library.bib}

\end{document}
